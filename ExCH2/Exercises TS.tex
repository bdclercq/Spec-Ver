\documentclass[12pt]{article}
\usepackage{graphicx} % This lets you include figures
\usepackage{hyperref} % This lets you make links to web locations
\graphicspath{ {./images/} }

\usepackage[rightcaption]{sidecap}
\usepackage{caption}
\usepackage{subcaption}
\usepackage{hyperref}

\usepackage{float}

\usepackage{imakeidx}

\usepackage{amsmath}

\makeindex


\title{Exercises on Transition Systems}
\author{Beau De Clercq}
\date{October 2019}

\begin{document}
\maketitle{}

\tableofcontents

\clearpage
\newpage

\section*{Ex. 2.1}
\subsection*{Question}
Consider the following two sequential hardware circuits:\\
\begin{centering}
	\includegraphics*[width=\linewidth]{circuits.png}
\end{centering}
\\
(a) Give the transition systems of both hardware circuits.\\
(b) Determine the reachable part of the transition system of the synchronous product of these
transition systems. Assume that the initial values of the registers are r1=0 and r2=1.

\subsection*{Answer}
(a) 
\begin{itemize}
	\item Circuit one:\\
	 \begin{centering}
		\includegraphics*[scale=0.5]{21a1.png}
	\end{centering}
	\item Circuit two:\\
	\begin{centering}
		\includegraphics*[scale=0.5]{21a2.png}
	\end{centering}
\end{itemize}
(b)
\begin{itemize}
	\item Reachable part:\\
	\begin{centering}
		\includegraphics*[scale=0.5]{21b.png}
	\end{centering}
	\item Not reachable: (00,00), (11,11), ...
\end{itemize}

\newpage

\section*{Ex. 2.2}
\subsection*{Question}
We are given three (primitive) processes $P_1, P_2,$ and $P_3$ with shared integer variable x. The program of process $P_i$ is as follows:\\
\begin{centering}
	\includegraphics*[scale=0.5]{algo1.png}
\end{centering}
\\
That is, $P_i$ executes ten times the assignment x := x+1. The assignment x := x+1 is realized using the three actions LOAD(x), INC(x) and STORE(x). Consider now the parallel program:\\
\begin{centering}
	\includegraphics*[scale=0.5]{algo2.png}
\end{centering}
\\
Does P have an execution that halts with the terminal value x = 2?

\subsection*{Answer}
The answer to this question depends on how you interpret the question.\\
\begin{itemize}
	\item If you allow processess to run completely independent from each other, it is possible to have an execution that halts with x = 2:
	\begin{itemize}
		\item First load x := 0 in $P_1, P_2, P_3$
		\item Then let $P_1$ finish (x == 10)
		\item After that, let $P_2$ run 9 times and store x = 9
		\item Increment $P_3$ 1 time and store
		\item Load x := 1 stored by $P_3$ in $P_2$ 
		\item Load x := 1 in $P_3$
		\item Finish $P_3$ before $P_2$ to end with x == 2
	\end{itemize}
	\item If you don't allow this and expect processes to wait until all others have finished their loop then it isn't possible.
\end{itemize}


\newpage
\section*{Ex. 2.3}
\subsection*{Question}
Consider the following street junction with the specification of a traffic light as
outlined on the right.\\
\begin{centering}
	\includegraphics*[width=\linewidth]{ex23.png}
\end{centering}
(a) Choose appropriate actions and label the transitions of the traffic light transition system
accordingly.\\
(b) Give the transition system representation of a (reasonable) controller C that switches the
green signal lamps in the following order: A1,A2,A3, A1,A2,A3, ... .\\
(Hint: Choose an appropriate communication mechanism.)\\
(c) Outline the transition system $A1\|A2\|A3\|C$.

\subsection*{Answer}
\begin{itemize}
	\item \begin{centering}
		\includegraphics*[width=\linewidth]{23a.png}
	\end{centering}
	\item \begin{centering}
		\includegraphics*[scale=0.5]{23b.png}
	\end{centering}
	\item 
\end{itemize}

\newpage
\section*{Ex. 2.4}
\subsection*{Question}
Show that the handshaking operator $\|$ that forces two transition systems to
synchronize over their common actions (see Definition 2.26 on page 48) is associative. That is,
show that
$(TS_1\|TS_2)\|TS_3 = TS_1\|(TS_2\|TS_3)$
where $TS_1, TS_2, TS_3$ are arbitrary transition systems.

\subsection*{Answer}
$\begin{array}{lcl}
	(TS_1\|TS_2)\|TS_3 & = & (S_1\times S_2, Act_1 \times Act_2, ->, I_1 \times I_2, AP_1\cup AP_2, L)\|TS_3\\
	 & = &(S_1 \times S_2 \times S_3, Act_1 \times Act_2 \times Act_3, ->, I_1 \times I_2 \times I_3, AP_1\cup AP_2 \cup AP_3, L)\\
	 & = & TS_1\|(S_2 \times S_3, Act_2 \times Act_3, ->, I_2 \times I_3, AP_2\cup AP_3, L)\\
	 & = & TS_1\|(TS_2\|TS_3)
\end{array}$

\newpage
\section*{Ex. 2.5}
\subsection*{Question}
The following program is a mutual exclusion protocol for two processes due to
Pnueli [118]. There is a single shared variable s which is either 0 or 1, and initially 1. Besides,
each process has a local Boolean variable y that initially equals 0. The program text for process
$P_i$ (i = 0, 1) is as follows:\\
\begin{centering}
	\includegraphics*[scale=1]{ex25.png}
\end{centering}\\
Here, the statement $(y_i, s) := (1, i)$; is a multiple assignment in which variable $y_i := 1$ and $s := i$
is a single, atomic step.\\
(a) Define the program graph of a process in Pnueli´s algorithm.\\
(b) Determine the transition system for each process.\\
(c) Construct their parallel composition.\\
(d) Check whether the algorithm ensures mutual exclusion.\\
(e) Check whether the algorithm ensures starvation freedom.\\

\subsection*{Answer}
\begin{itemize}
	\item \begin{centering}
		\includegraphics*[scale=0.5]{25a.png}
	\end{centering}
	\item \begin{centering}
		\includegraphics*[scale=0.5]{25b.png}
	\end{centering}
	\item \begin{centering}
		\includegraphics*[scale=0.5]{25c.png}
	\end{centering}
	\item P = $\{CRIT_1, CRIT_2, r_1, r_2\}$\\
		ME is ensured:
	\item Starvation freedom is not ensured:
\end{itemize}

\newpage
\section*{Ex. 2.6}
\subsection*{Question}
Consider a stack of nonnegative integers with capacity n (for some fixed n).\\
(a) Give a transition system representation of this stack. You may abstract from the values on
the stack and use the operations top, pop, and push with their usual meaning.\\
(b) Sketch a transition system representation of the stack in which the concrete stack content is
explicitly represented.

\subsection*{Answer}
\begin{itemize}
	\item \begin{centering}
		\includegraphics*[scale=0.7]{26a.png}
	\end{centering}
	\item \begin{centering}
		\includegraphics*[scale=0.55]{26b.png}
	\end{centering}
\end{itemize}

\newpage
\section*{Ex. 2.7}
\subsection*{Question}
Consider the following generalization of Peterson’s mutual exclusion algorithm
that is aimed at an arbitrary number n (n $\geq$ 2) processes.\\
The basic concept of the algorithm is that each process passes through n “levels” before acquiring access to the critical section. The concurrent processes share the bounded integer arrays y[0..n−1] and p[1..n] with y[i] ∈ {1, . . . , n } and p[i] ∈ {0, . . . , n−1 }. \\
y[j] = i means that process i has the lowest priority at level j, and p[i] = j expresses that process i is currently at level j. Process i starts at level 0. \\
On requesting access to the critical section, the process passes through levels 1 through n−1. Process i waits at
level j until either all other processes are at a lower level (i.e., p[k] $<$ j for all k $\neq$ i) or another
process grants process i access to its critical section (i.e., y[j] $\neq$ i). The behavior of process i is in
pseudocode:\\
\begin{centering}
	\includegraphics*[scale=0.8]{ex27.png}
\end{centering}\\
(a) Give the program graph for process i.\\
(b) Determine the number of states (including the unreachable states) in the parallel composition
of n processes.\\
(c) Prove that this algorithm ensures mutual exclusion for n processes.\\
(d) Prove that it is impossible that all processes are waiting in the for-iteration.\\
(e) Establish whether it is possible that a process that wants to enter the critical section waits
ad infinitum.\\

\subsection*{Answer}

\newpage
\section*{Ex. 2.8}
\subsection*{Question}

\subsection*{Answer}

\newpage
\section*{Ex. 2.9}
\subsection*{Question}

\subsection*{Answer}

\newpage
\section*{Ex. 2.10}
\subsection*{Question}

\subsection*{Answer}

\newpage
\section*{Ex. 2.11}
\subsection*{Question}

\subsection*{Answer}

\end{document}