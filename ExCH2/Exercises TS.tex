\documentclass[12pt]{article}
\usepackage{graphicx} % This lets you include figures
\usepackage{hyperref} % This lets you make links to web locations
\graphicspath{ {./images/} }

\usepackage[rightcaption]{sidecap}
\usepackage{caption}
\usepackage{subcaption}
\usepackage{hyperref}

\usepackage{float}

\usepackage{imakeidx}

\makeindex


\title{Exercises on Transition Systems}
\author{Beau De Clercq}
\date{October 2019}

\begin{document}
\maketitle{}

\tableofcontents

\clearpage
\newpage

\section*{Ex. 2.1}
\subsection*{Question}
Consider the following two sequential hardware circuits:\\
\begin{centering}
	\includegraphics*[width=\linewidth]{circuits.png}
\end{centering}
\\
(a) Give the transition systems of both hardware circuits.\\
(b) Determine the reachable part of the transition system of the synchronous product of these
transition systems. Assume that the initial values of the registers are r1=0 and r2=1.

\subsection*{Answer}

\newpage
\section*{Ex. 2.2}
\subsection*{Question}
We are given three (primitive) processes P1,P2, and P3 with shared integer variable x. The program of process Piis as follows:\\
\begin{centering}
	\includegraphics*[width=\linewidth]{algo1.png}
\end{centering}
\\
That is, Pi executes ten times the assignment x := x+1. The assignment x := x+1 is realized using the three actions LOAD(x), INC(x) and STORE(x). Consider now the parallel program:\\
\begin{centering}
	\includegraphics*[width=\linewidth]{algo2.png}
\end{centering}
\\
Does P have an execution that halts with the terminal value x = 2?

\subsection*{Answer}

\newpage
\section*{Ex. 2.3}
\subsection*{Question}

\subsection*{Answer}

\newpage
\section*{Ex. 2.4}
\subsection*{Question}

\subsection*{Answer}

\newpage
\section*{Ex. 2.5}
\subsection*{Question}

\subsection*{Answer}

\newpage
\section*{Ex. 2.6}
\subsection*{Question}

\subsection*{Answer}

\newpage
\section*{Ex. 2.7}
\subsection*{Question}

\subsection*{Answer}

\newpage
\section*{Ex. 2.8}
\subsection*{Question}

\subsection*{Answer}

\newpage
\section*{Ex. 2.9}
\subsection*{Question}

\subsection*{Answer}

\newpage
\section*{Ex. 2.10}
\subsection*{Question}

\subsection*{Answer}

\newpage
\section*{Ex. 2.11}
\subsection*{Question}

\subsection*{Answer}

\end{document}