\documentclass[12pt]{article}
\usepackage{graphicx} % This lets you include figures
\usepackage{hyperref} % This lets you make links to web locations
\graphicspath{ {./images/} }

\usepackage[rightcaption]{sidecap}
\usepackage{caption}
\usepackage{subcaption}
\usepackage{hyperref}

\usepackage{float}

\usepackage{imakeidx}

\usepackage{amsmath}
\usepackage{latexsym}
\usepackage{wasysym}

\usepackage{color,soul}
\usepackage{listings}

\definecolor{darkgray}{rgb}{.4,.4,.4}
\definecolor{purple}{rgb}{0.65, 0.12, 0.82}

\lstdefinelanguage{PRISM}{
	keywords={module, init, endinit, endmodule, dtmc},
	keywordstyle=\color{purple}\bfseries,
	ndkeywords={bool},
	ndkeywordstyle=\color{darkgray}\bfseries,
	identifierstyle=\color{red},
	sensitive=false,
	comment=[l]{//},
	morecomment=[s]{/*}{*/},
	commentstyle=\color{purple}\ttfamily
}

\newcommand\tab[1][1cm]{\hspace*{#1}}

\makeindex


\title{Exercises on Chapter 6}
\author{Beau De Clercq}
\date{November 2019}

\begin{document}
\maketitle{}

%\tableofcontents

\clearpage
\newpage

\section*{Ex. 6.1}
\subsection*{Question}
Consider the following transition system over AP = $\{ b, g, r, y \}$:
\begin{center}
	\includegraphics*[scale=0.8]{ex61.png}
\end{center}
The following atomic propositions are used: $r$ (red), $y$ (yellow), $g$ (green), and $b$ (black). The
model is intended to describe a traffic light that is able to blink yellow. You are requested to
indicate for each of the following CTL formulae the set of states for which these formulae hold:
\begin{center}
	\includegraphics*[scale=0.8]{ex61statements.png}
\end{center}

\subsection*{Answer}
\begin{itemize}
	\item $\{1,2,3,4\}$
	\item $\{\}$
	\item $\{1,2,3,4\}$
	\item $\{1,3\}$
	\item $\{1,2,3,4\}$
	\item $\{\}$
	\item $\{2,4\}$
	\item $\{1,2,3,4\}$
	\item $\{1,2,3,4\}$
	\item $\{1,2,3,4\}$
	\item $\{1\}$
	\item $\{4\}$
\end{itemize}


\newpage
\section*{Ex. 6.2}
\subsection*{Question}
Consider the following CTL formulae and the transition system $TS$ outlined on
the right:
\begin{center}
	\includegraphics*[scale=0.6]{ex62.png}
\end{center}
Determine the satisfaction sets Sat($\phi_i$) and decide whether $TS \models\phi_i (1\leq i\leq 4)$.

\subsection*{Answer}
\begin{itemize}
	\item $\phi_1$: $TS \not\models \phi_1$ (checked with PRISM), Sat($\phi_1$) = $\{s_0, s_1\}$
	\item $\phi_2$: $TS \not\models \phi_2$ (checked with PRISM), Sat($\phi_2$) = $\{s_1, s_2, s_3\}$
	\item $\phi_3$: $TS \models \phi_3$ (checked with PRISM), Sat($\phi_3$) = $\{s_0, s_2, s_3\}$
	\item $\phi_4$: $TS \models \phi_4$ (checked with PRISM), Sat($\phi_4$) = $\{s_0, s_2, s_3\}$
\end{itemize}

\newpage
\section*{Ex. 6.3}
\subsection*{Question}
Which of the following assertions are correct? Provide a proof or a counterexample.
\begin{itemize}
	\item If $s\models\exists\Square a$, then $s\models\forall\Square a$.
	\item If $s\models\forall\Square a$, then $s\models\exists\Square a$.
	\item If $s\models\forall\Diamond a\vee\forall\Diamond b$, then $s\models\forall\Diamond(a\vee b)$.
	\item If $s\models\forall\Diamond(a\vee b)$, then $s\models\forall\Diamond a\vee \forall\Diamond b$.
	\item If $s\models\forall(aUb)$, then $s\models\neg(\exists(\neg bU(\neg a\wedge\neg b))\vee\exists\Square\neg b)$.
\end{itemize}

\subsection*{Answer}
\begin{itemize}
	\item Incorrect: the first $\models$ allows paths that not always see a while the second one only contains such paths (so if $s \models expr_1$ then $s \not\models expr_2$).
	\item Correct: $expr_1$ always sees a on all paths and $expr_2$ states that there are paths that always see a, so if $s\models expr_1$ it also holds that $s\models expr_2$.
	\item Correct: both expressions have the same meaning.
	\item Correct: both expressions have the same meaning.
	\item Correct: both expressions have the same meaning.
\end{itemize}

\newpage
\section*{Ex. 6.4}
\subsection*{Question}
Let $\phi$ and $\psi$ be arbitrary CTL formulae. Which of the following equivalences for CTL formulae are correct?
\begin{center}
	\includegraphics*[scale=0.8]{ex64.png}
\end{center}
\subsection*{Answer}
\begin{itemize}
	\item Correct
	\item Correct
	\item Incorrect
	\item Correct
	\item Incorrect
	\item Incorrect
	\item Incorrect
	\item Correct
	\item Incorrect
	\item Incorrect
	\item Correct
	\item Incorrect
	\item Correct
	\item Incorrect
	\item Correct
\end{itemize}

\newpage
\section*{Ex. 6.7}
\subsection*{Question}
Transform the following CTL formulae into ENF and PNF. Show all intermediate
steps.
\begin{center}
	\includegraphics*[scale=0.8]{ex67.png}
\end{center}

\subsection*{Answer}
\begin{itemize}
	\item ENF: $\neg\exists(((\neg a)\wedge\neg(b\implies\neg\exists\Circle\neg c)U(\neg(\neg a)\wedge\neg(b\implies\neg\exists\Circle\neg c))))$\\
		PNF: allready in PNF
	\item ENF: $\neg\exists\Circle\neg(\exists((\neg a)U(b\wedge\neg c))\vee\exists\Square\neg\exists\Circle\neg a$\\
		PNF: allready in PNF
\end{itemize}

\newpage
\section*{Ex. 6.8}
\subsection*{Question}
Provide two finite transition systems $TS_1$ and $TS_2$ (without terminal states, and over the same set of atomic propositions) and a CTL formula $\phi$ such that Traces($TS_1$) = Traces($TS_2$) and $TS_1 \models\phi$, but $TS_2 \not\models\phi$.

\subsection*{Answer}
\begin{center}
	\includegraphics*[scale=0.8]{ex681.png}
	\includegraphics*[scale=0.8]{ex682.png}
\end{center}

\section*{Ex. 6.9}
\subsection*{Question}

\subsection*{Answer}

\section*{Ex. 6.13}
\subsection*{Question}

\subsection*{Answer}

\newpage
\section*{Ex. 6.14}
\subsection*{Question}
Check for each of the following formula pairs $(\phi_i, \varphi_i)$ whether the CTL formula $\phi_i$ is equivalent to the LTL formula $\varphi_i$. Prove the equivalence or provide a counterexample that illustrates why $\phi_i \not\equiv \varphi_i$.
\begin{center}
	\includegraphics*[scale=0.8]{ex614.png}
\end{center}

\subsection*{Answer}
\begin{itemize}
	\item Equivalent (absorption law)
	\item Equivalent (absortpion law)
	\item Equivalent: the $\forall$ in $\phi$ can be dropped (because of the $\Diamond$) which leaves $\Diamond(a\wedge\exists\Circle a) \equiv \Diamond(a\wedge\Circle a)$. Both expressions state there needs to be a path that has $a$ followed by another $a$.
	\item Equivalent: stating that for all paths you will eventually see a or eventually see b is the same as stating that eventually you will see a or b.
	\item Equivalent: $\phi_5$ states that for all paths it always holds that seeing an a implies that you will always eventually see a b. This is the same meaning $\varphi_5$ has in LTL.
	\item Not equivalent: $\phi_6$ only contains paths $bbbbb.....abbbbbbbbb....$ where $\varphi_6$ also satisfies $\neg b\neg b ... \neg babbbbbbb.....$.
\end{itemize}


\section{Appendix}
\subsection{PRISM code (ex 6.2)}
\begin{lstlisting}[language=PRISM]
dtmc

module ex2
// local state
s : [0..4];
a : bool;
b : bool;

[] s=0 -> (s'=1) & (a'=true);
[] s=0 -> (s'=4) & (b'=true);
[] s=1 -> (s'=2) & (a'=true);
[] s=1 -> (s'=2) & (b'=true);
[] s=2 -> (s'=3) & (b'=true) & (a'=false);
[] s=3 -> (s'=3) & (b'=true);
[] s=3 -> (s'=0) & (b'=false);
[] s=4 -> (s'=4) & (b'=true);

endmodule

init
(s=0 | s=3) & a=false & b=false
endinit
\end{lstlisting}

\subsection{PRISM properties (ex 6.2)}
\begin{lstlisting}[language=PRISM]
(A[(a=true)U(b=true)]) | (E[X(A[G(b=true)])])

A[G(A[(a=true)U(b=true)])]

((a=true) & (b=true)) => E[G(E[X(A[(b=true)W(a=true)])])]

A[G(E[F(((a=true) & (b=true)) => E[G(E[X(A[(b=true)W(a=true)])])])])]

\end{lstlisting}



\end{document}