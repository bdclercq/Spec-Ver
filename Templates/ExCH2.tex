\documentclass[12pt]{article}
\usepackage{graphicx} % This lets you include figures
\usepackage{hyperref} % This lets you make links to web locations
\graphicspath{ {./images/} }

\usepackage[rightcaption]{sidecap}
\usepackage{caption}
\usepackage{subcaption}
\usepackage{hyperref}

\usepackage{float}

\usepackage{imakeidx}

\usepackage{amsmath}

\makeindex


\title{Exercises on Transition Systems}
\author{Beau De Clercq}
\date{October 2019}

\begin{document}
\maketitle{}

%\tableofcontents

\clearpage
\newpage

\section*{Ex. 2.1}
\subsection*{Question}
Consider the following two sequential hardware circuits:\\
\begin{centering}
	\includegraphics*[width=\linewidth]{circuits.png}
\end{centering}
\\
(a) Give the transition systems of both hardware circuits.\\
(b) Determine the reachable part of the transition system of the synchronous product of these
transition systems. Assume that the initial values of the registers are r1=0 and r2=1.

\subsection*{Answer}

\newpage

\section*{Ex. 2.2}
\subsection*{Question}
We are given three (primitive) processes $P_1, P_2,$ and $P_3$ with shared integer variable x. The program of process $P_i$ is as follows:\\
\begin{centering}
	\includegraphics*[scale=0.5]{algo1.png}
\end{centering}
\\
That is, $P_i$ executes ten times the assignment x := x+1. The assignment x := x+1 is realized using the three actions LOAD(x), INC(x) and STORE(x). Consider now the parallel program:\\
\begin{centering}
	\includegraphics*[scale=0.5]{algo2.png}
\end{centering}
\\
Does P have an execution that halts with the terminal value x = 2?

\subsection*{Answer}

\newpage
\section*{Ex. 2.3}
\subsection*{Question}
Consider the following street junction with the specification of a traffic light as
outlined on the right.\\
\begin{centering}
	\includegraphics*[width=\linewidth]{ex23.png}
\end{centering}
(a) Choose appropriate actions and label the transitions of the traffic light transition system
accordingly.\\
(b) Give the transition system representation of a (reasonable) controller C that switches the
green signal lamps in the following order: A1,A2,A3, A1,A2,A3, ... .\\
(Hint: Choose an appropriate communication mechanism.)\\
(c) Outline the transition system $A1\|A2\|A3\|C$.

\subsection*{Answer}

\newpage
\section*{Ex. 2.4}
\subsection*{Question}
Show that the handshaking operator $\|$ that forces two transition systems to
synchronize over their common actions (see Definition 2.26 on page 48) is associative. That is,
show that
$(TS_1\|TS_2)\|TS_3 = TS_1\|(TS_2\|TS_3)$
where $TS_1, TS_2, TS_3$ are arbitrary transition systems.

\subsection*{Answer}

\newpage
\section*{Ex. 2.5}
\subsection*{Question}
The following program is a mutual exclusion protocol for two processes due to
Pnueli [118]. There is a single shared variable s which is either 0 or 1, and initially 1. Besides,
each process has a local Boolean variable y that initially equals 0. The program text for process
$P_i$ (i = 0, 1) is as follows:\\
\begin{centering}
	\includegraphics*[scale=1]{ex25.png}
\end{centering}\\
Here, the statement $(y_i, s) := (1, i)$; is a multiple assignment in which variable $y_i := 1$ and $s := i$
is a single, atomic step.\\
(a) Define the program graph of a process in Pnueli´s algorithm.\\
(b) Determine the transition system for each process.\\
(c) Construct their parallel composition.\\
(d) Check whether the algorithm ensures mutual exclusion.\\
(e) Check whether the algorithm ensures starvation freedom.\\

\subsection*{Answer}

\newpage
\section*{Ex. 2.6}
\subsection*{Question}
Consider a stack of nonnegative integers with capacity n (for some fixed n).\\
(a) Give a transition system representation of this stack. You may abstract from the values on
the stack and use the operations top, pop, and push with their usual meaning.\\
(b) Sketch a transition system representation of the stack in which the concrete stack content is
explicitly represented.

\subsection*{Answer}

\newpage
\section*{Ex. 2.7}
\subsection*{Question}
Consider the following generalization of Peterson’s mutual exclusion algorithm
that is aimed at an arbitrary number n (n $\geq$ 2) processes.\\
The basic concept of the algorithm is that each process passes through n “levels” before acquiring access to the critical section. The concurrent processes share the bounded integer arrays y[0..n−1] and p[1..n] with y[i] ∈ {1, . . . , n } and p[i] ∈ {0, . . . , n−1 }. \\
y[j] = i means that process i has the lowest priority at level j, and p[i] = j expresses that process i is currently at level j. Process i starts at level 0. \\
On requesting access to the critical section, the process passes through levels 1 through n−1. Process i waits at
level j until either all other processes are at a lower level (i.e., p[k] $<$ j for all k $\neq$ i) or another
process grants process i access to its critical section (i.e., y[j] $\neq$ i). The behavior of process i is in
pseudocode:\\
\begin{centering}
	\includegraphics*[scale=0.8]{ex27.png}
\end{centering}\\
(a) Give the program graph for process i.\\
(b) Determine the number of states (including the unreachable states) in the parallel composition
of n processes.\\
(c) Prove that this algorithm ensures mutual exclusion for n processes.\\
(d) Prove that it is impossible that all processes are waiting in the for-iteration.\\
(e) Establish whether it is possible that a process that wants to enter the critical section waits
ad infinitum.\\

\subsection*{Answer}

\newpage
\section*{Ex. 2.8}
\subsection*{Question}
In channel systems, values can be transferred from one process to another process.\\
As this is somewhat limited, we consider in this exercise an extension that allows for the transfer
of expressions.\\
That is to say, the send and receive statements c!v and c?x (where x and v are of
the same type) are generalized into c!expr and c?x, where for simplicity it is assumed that expr
is a correctly typed expression (of the same type as x). Legal expressions are, e.g., x $\wedge$ ($\neg$y $\lor$ z)
for Boolean variables x, y, and z, and channel c with dom(c) = $\{$ 0, 1 $\}$. For integers x, y, and an
integer-channel c, $|$2x + (x − y)div17$|$ is a legal expression.\\
\\
Extend the transition system semantics of channel systems such that expressions are
allowed in send statements.

\subsection*{Answer}

\newpage
\section*{Ex. 2.9}
\subsection*{Question}
Consider the following mutual exclusion algorithm that uses the shared variables
y1 and y2 (initially both 0).\\
\begin{centering}
	\includegraphics*[scale=1]{ex29.png}
\end{centering}
\\
(a) Give the program graph representations of both processes. (A pictorial representation suffices.)\\
(b) Give the reachable part of the transition system of P1 $\|$ P2 where y1 $\leq$ 2 and y2 $\leq$ 2.\\
(c) Describe an execution that shows that the entire transition system is infinite.\\
(d) Check whether the algorithm indeed ensures mutual exclusion.\\
(e) Check whether the algorithm never reaches a state in which both processes are mutually
waiting for each other.\\
(f) Is it possible that a process that wants to enter the critical section has to wait ad infinitum?\\

\subsection*{Answer}

\newpage
\section*{Ex. 2.10}
\subsection*{Question}
Consider the following mutual exclusion algorithm that was proposed 1966 [221]
as a simplification of Dijkstra’s mutual exclusion algorithm in case there are just two processes:\\
\begin{centering}
	\includegraphics*[scale=0.8]{ex210.png}
\end{centering}\\
Here C0, C1, and C2 are program labels, and the word “computer” should be interpreted as process.\\
\\
(a) Give the program graph representations for a single process. (A pictorial representation
suffices.)\\
(b) Give the reachable part of the transition system of P1 $\|$ P2.\\
(c) Check whether the algorithm indeed ensures mutual exclusion.

\subsection*{Answer}

\newpage
\section*{Ex. 2.11}
\subsection*{Question}
Consider the following two sequential hardware circuits C1 and C2:\\
\begin{centering}
	\includegraphics*[scale=0.8]{ex211.png}
\end{centering}\\
\\
(a) Give the transition system representation TS(C1) of the circuit C1.\\
(b) Let TS(C2) be the transition system of the circuit C2. Outline the transition system TS(C1) $\otimes$
TS(C2).

\subsection*{Answer}

\end{document}