\documentclass[12pt]{article}
\usepackage{graphicx} % This lets you include figures
\usepackage{hyperref} % This lets you make links to web locations
\graphicspath{ {./images/} }

\usepackage[rightcaption]{sidecap}
\usepackage{caption}
\usepackage{subcaption}
\usepackage{hyperref}

\usepackage{float}

\usepackage{imakeidx}

\usepackage{amsmath}
\usepackage{latexsym}
\usepackage{wasysym}

\usepackage{color,soul}
\usepackage{listings}


\newcommand\tab[1][1cm]{\hspace*{#1}}

\makeindex


\title{Pseudo-code}
\author{Beau De Clercq}
\date{November 2019}

\begin{document}
\maketitle{}

%\tableofcontents


\section{Ex. 1 p6}
\subsection{Question}
What is the computational complexity of determining whether u can reach v in a given graph? Can you describe an efficient algorithm to solve this problem?

\subsection{Answer}
Strongly Connected Components:
\begin{itemize}
	\item Perform a DFS on the graph $G = (V, E)$.
	\item Construct the graph $G' = (V, E')$ where $(u, v) \in E'$ iff $(v, u) \in E$.
	\item Perform a DFS on $G'$ (the SCC's will be produced in this step).
\end{itemize}
Complexity: $O(|V|+|E|)$.

\newpage
\section{Ex. 2 p8}
\subsection{Question}
Is the language L = $\{w|\#_a(w) \geq \#b(w)\}$, where $\#_\sigma(w)$ denotes the number of times the symbol $\sigma$ occurs in $w$, an $\omega$-regular language?

\subsection{Answer}
No: counting is not possible.


\section{Ex. 3 p9}
\subsection{Question}
Are the words $a^\omega$ and $ab^\omega$ in the language of the automaton from Figure 1 in the following cases?
\begin{itemize}
	\item with reachability acceptance condition and T = $\{q_0\}$
	\item with safety acceptance condition and U = $\{q_2\}$
	\item with Büchi acceptance condition and B = $\{q_2\}$
	\item with co-Büchi acceptance condition and B = $\{q_2\}$
	\item with parity acceptance condition
\end{itemize}

\subsection{Answer}


\section{Ex. 4 p11}
\subsection{Question}
Consider the sequence $\alpha = 1,-1, 1, 1,-1, 1, 1, 1, 1, . . .$ where the i-th $-1$ is followed by $2^i$ occurrences of 1. What are the values of $\textrm{\underline{MP(a)}}$ and $\overline{MP(a)}$?

\subsection{Answer}


\section{Ex. 5 p12}
\subsection{Question}
With $\aleph = 3^\omega$ and $\lambda = \frac{3}{4}$, what is the value of DS$_\lambda(\aleph)$?

\subsection{Answer}


\section{Ex. 6 p12}
\subsection{Question}
Using the meanpayoff example, prove that the parity and co-Büchi payoff
 functions are Borel when $\triangleright = \geq$ and $a = 1$.

\subsection{Answer}


\section{Ex. 7 p16}
\subsection{Question}
In the game from Figure 2, does Eve have a strategy to ensure her mean-payoff value is non-negative?

\subsection{Answer}


\section{Ex. 8 p16}
\subsection{Question}
Describe the Mealy machine
that corresponds to the strategy Eve for
the game in Figure 2 which consists in
playing from $v_0$ to $v_1$ every other time
$v_0$ is visited and to $v_0$ otherwise.

\subsection{Answer}


\section{Ex. 9 p17}
\subsection{Question}
Describe the product of the strategy from the previous exercise and the game.

\subsection{Answer}


\section{Ex. 10 p21}
\subsection{Question}
Is ``staying in $v_0$ forever''
a worst-case optimal strategy for Eve
in the game from Figure 2 with the
mean-payoff function?

\subsection{Answer}


\section{Ex. 11 p21}
\subsection{Question}
Describe best-case optimal strategies for both players in the game from Figure 2 with the mean-payoff function. What is the co-operative value of the game?

\subsection{Answer}


\section{Ex. 12 p22}
\subsection{Question}
Consider the game from Figure 4 and suppose that Adam wants to ensure at most two vertices distinct are visited. From which vertices can he win against Eve? Describe a strategy of his that witnesses the fact. What if we allow at most three vertices?

\subsection{Answer}


\section{Ex. 13 p27}
\subsection{Question}
Give a proof by induction of Theorem 10 for Eve based on the definition of the attractor sets.

\subsection{Answer}


\section{Ex. 14 p28}
\subsection{Question}
Prove Theorem 11.

\subsection{Answer}


\section{Ex. 15 p28}
\subsection{Question}
Prove Theorem 12 using the fact that positional strategies suffice for both players in reachability games.

\subsection{Answer}


\section{Ex. 16 p28}
\subsection{Question}
Using the fact that $R$ is exactly the set of vertices from which a player wins a reachability game (and the other one loses a safety game), prove that $G\setminus R$ contains no sinks. Prove that $G\setminus S$ also contains no sinks.

\subsection{Answer}


\section{Ex. 17 p29}
\subsection{Question}
Prove the upper bound on the running time of the algorithm.

\subsection{Answer}

\section{Ex. 18 p30}
\subsection{Question}
Given a (co-)Büchi game, construct a parity game with the same arena such that both players win from a vertex for the original objective if and only if they win from it for the parity objective.
(Tip: you should not use more than three priorities)

\subsection{Answer}

\section{Ex. 19}
\subsection{Question}
Based on how many recursive calls Zielonka’s algorithm makes on (co-)Büchi games, conclude that determining the winner of such games is decidable in polynomial time.

\subsection{Answer}

\section{Ex. 20}
\subsection{Question}
Compute all distinct attractor sets for both players in the game from Figure 5. From which vertices can each player guarantee to win the game?

\subsection{Answer}

\section{Ex. 21}
\subsection{Question}
Run Zielonka’s algorithm to determine the vertices from which each player can guarantee to win the parity game from Figure 6.

\subsection{Answer}



\end{document}